\documentclass[conference]{IEEEtran}
\IEEEoverridecommandlockouts

\usepackage{cite}
\usepackage{amsmath,amssymb,amsfonts}
\usepackage{algorithmic}
\usepackage{graphicx}
\usepackage{textcomp}
\usepackage{xcolor}
\usepackage{hyperref}
\usepackage{listings}
\usepackage{booktabs}
\usepackage{multirow}
\usepackage{subfigure}

\def\BibTeX{{\rm B\kern-.05em{\sc i\kern-.025em b}\kern-.08em
    T\kern-.1667em\lower.7ex\hbox{E}\kern-.125emX}}

\begin{document}

\title{CodeSage: A Local Large Language Model Assisted Runtime Intelligence Layer with Model Context Protocol for Self Healing Application Errors in .NET Environments Using LM Studio API}

\author{\IEEEauthorblockN{Mateus Yonathan}
\IEEEauthorblockA{Independent Researcher\\
LinkedIn: \href{https://www.linkedin.com/in/siyoyo/}{siyoyo}}

\maketitle

\begin{abstract}
This paper presents CodeSage, a novel runtime middleware layer for .NET 9 applications that enhances software reliability through local Large Language Model (LLM) assistance and Model Context Protocol (MCP). CodeSage leverages the LM Studio API to provide real-time, privacy-preserving error analysis and self-healing capabilities. By intercepting unhandled exceptions and generating natural language explanations with actionable suggestions, CodeSage introduces an intelligent AI layer that operates fully offline while maintaining interoperability through standardized MCP communication. The system's architecture combines exception interception middleware, dynamic context generation, and local LLM inference to provide immediate, context-aware support for runtime error resolution. This research demonstrates the practical impact of combining local LLMs with MCP for runtime application error handling, offering a new paradigm in software reliability that shifts AI-assisted debugging from compile-time to runtime while addressing privacy and connectivity constraints critical for edge and enterprise environments.
\end{abstract}

\begin{IEEEkeywords}
Runtime Intelligence, Large Language Models, Model Context Protocol, Self-healing Systems, .NET, LM Studio, Error Handling, Middleware
\end{IEEEkeywords}

\section{Introduction}
\section{Introduction}
Modern software applications, especially complex and distributed systems, face significant challenges in effectively handling runtime errors~\cite{dist_systems_errors_2018, microservices_challenges_2020, runtime_error_analysis_2023, error_handling_survey_2024}. Traditional error management strategies, relying primarily on static analysis, detailed logging, and manual debugging, are often insufficient to address the dynamic and intricate nature of errors encountered in production environments~\cite{debugging_complex_systems_2021, runtime_debugging_challenges_2023, error_recovery_survey_2024}. These methods frequently lead to prolonged downtime, increased operational costs, and a suboptimal user experience due to delayed error identification and resolution.

Recent advancements in Large Language Models (LLMs) have demonstrated remarkable capabilities in understanding and generating code, opening new avenues for automated software engineering tasks, including code analysis and debugging support~\cite{llm_code_generation_2022, llm_debugging_2023, llm_software_engineering_2024, local_llm_applications_2024}. However, applying large, powerful LLMs directly to real time runtime error analysis in sensitive or resource constrained environments presents its own set of challenges. Privacy concerns associated with transmitting potentially sensitive runtime data to external services, the need for low latency responses for real time remediation, and the dependency on stable network connectivity limit the applicability of cloud hosted LLMs in many scenarios~\cite{privacy_llm_challenges_2023, edge_ai_challenges_2019, local_llm_security_2024, runtime_llm_optimization_2024}.

RuntimeErrorSage addresses these critical limitations by proposing and implementing a runtime middleware system that leverages a local Large Language Model for intelligent error analysis and automated remediation. Operating entirely offline, RuntimeErrorSage utilizes a standard HTTP API interface to interact with a locally hosted LLM, specifically the Qwen 2.5 7B Instruct 1M model. This approach ensures data privacy, minimizes latency, and provides a robust solution independent of external network dependencies, making it particularly suitable for enterprise applications, edge deployments, and environments with strict data governance policies.

Our work makes the following key contributions:
\begin{itemize}
\item We introduce RuntimeErrorSage, a system architecture for intelligent runtime error analysis and automated remediation utilizing a local LLM.
\item We present a formal mathematical framework encompassing models for runtime error classification, context management, and remediation decision making.
\item We detail the implementation of a .NET middleware layer for real time error interception and processing.
\item We provide a comprehensive evaluation demonstrating the system's effectiveness in terms of error classification accuracy, remediation success rate, and runtime overhead.
\item We show that leveraging a local, instruct tuned LLM (Qwen 2.5 7B Instruct 1M) via a standard API enables performant and privacy preserving runtime error handling.
\item The integration of AI techniques, particularly machine learning, has been explored to enhance the diagnostic and planning capabilities of self healing systems, but leveraging the natural language understanding and reasoning abilities of LLMs for complex error scenarios represents a newer direction.
\end{itemize}

RuntimeErrorSage distinguishes itself from existing work by combining the strengths of local LLM inference, advanced context management, and a formal system model within a practical middleware architecture for automated runtime error remediation. Unlike systems relying on external services or predefined recovery strategies, our system offers a privacy preserving, low latency, and intelligent approach to handling a wide range of runtime errors, including those not previously encountered.

The current implementation of RuntimeErrorSage includes a robust exception handling system with context-aware error tracking, ASP.NET Core middleware for exception interception, and standardized error response models. The system demonstrates practical capabilities in handling common runtime errors through example endpoints for database operations, file management, service integration, and resource allocation.

RuntimeErrorSage's architecture is designed to intercept unhandled exceptions during application execution, generate rich contextual information, and leverage local LLM inference to provide natural language explanations and remediation suggestions. The system operates fully offline, addressing critical privacy and connectivity constraints while maintaining interoperability through the MCP framework.

The source code for RuntimeErrorSage, including the implementation of the middleware layer, LM Studio integration, and Model Context Protocol, is available as open-source software at \url{https://github.com/myonathanlinkedin/paper_research}.

The remainder of this paper is organized as follows: Section~\ref{sec:related-work} reviews related work in AI-assisted programming, static analysis, and runtime error handling. Section~\ref{sec:scope} presents the scope of this research. Section~\ref{sec:implementation} describes the implementation details, including the middleware components and LLM integration. Section~\ref{sec:case-studies} presents case studies demonstrating RuntimeErrorSage's effectiveness. Section~\ref{sec:evaluation} evaluates the system's performance and accuracy. Section~\ref{sec:conclusion} discusses limitations and concludes the paper.

\section{Related Work}
\section{Related Work}\label{sec:related-work}
Recent advances in Large Language Models (LLMs) have revolutionized software development practices.

\subsection{Runtime Error Analysis}
Traditional approaches to runtime error analysis primarily rely on manual inspection of logs and debugging tools, static code analysis to identify potential issues before runtime, and post mortem analysis of crash dumps~\cite{debugging_techniques_survey_2017, static_analysis_overview_2015}. While effective for certain types of errors, these methods often struggle with dynamic runtime phenomena, complex interactions in distributed systems, and require significant human effort and expertise. 

Automated log analysis techniques~\cite{log_analysis_survey_2016} have been developed to process large volumes of log data, but they typically depend on predefined patterns and lack the ability to reason about error scenarios or system specific context without explicit programming.

\subsection{Self Healing Systems}
Research into self healing or autonomic computing systems has explored architectures and mechanisms for software systems to detect, diagnose, and recover from failures autonomously~\cite{autonomic_computing_overview_2004, self_healing_survey_2012}. These systems often employ feedback loops, such as the Monitor Analyze Plan Execute Knowledge (MAPE-K) loop~\cite{mapek2003}, to manage their own behavior and adapt to changing conditions or failures.

Remediation strategies in these systems can range from simple restarts and reconfigurations to more complex state rollbacks or dynamic code updates. However, many existing self healing solutions require significant a priori knowledge about potential failure modes and corresponding recovery actions, limiting their effectiveness against unforeseen errors. The integration of AI techniques, particularly machine learning, has been explored to enhance the diagnostic and planning capabilities of self healing systems, but leveraging the natural language understanding and reasoning abilities of LLMs for complex error scenarios represents a newer direction.

\subsection{Context Aware Computing and Debugging}
Context aware computing focuses on systems that can perceive their environment and adapt their behavior based on contextual information~\cite{context_aware_computing_survey_2009}. In the realm of software engineering and debugging, context awareness involves utilizing information about the system's state, execution environment, user interactions, and history to aid in understanding and resolving issues~\cite{context-aware-debugging-2023}.

Techniques include dynamic slicing, state tracing, and environmental monitoring to gather relevant context. While these techniques are powerful for providing visibility into the system, the challenge remains in effectively processing and reasoning about potentially vast and complex contextual data to pinpoint the root cause of an error and devise an appropriate solution. Our work utilizes context management techniques but enhances the analysis capabilities by feeding this context into a powerful LLM.

\subsection{Large Language Models in Software Engineering}
Large Language Models have rapidly emerged as powerful tools for a variety of software engineering tasks, including code completion~\cite{copilot2021}, code generation~\cite{llm_code_generation_2022}, code summarization, and vulnerability detection~\cite{llm_security_applications_2023}. Their ability to understand and generate human language and code has opened possibilities for more intelligent automated tools.

However, directly applying general purpose LLMs to real time, performance critical tasks like runtime error remediation requires careful consideration of latency, cost, and data privacy. The use of smaller, specialized, or locally hosted models is an active area of research to address these challenges~\cite{local_llm_deployment_2023, edge_llm_inference_2022}. Our approach specifically investigates the practical application of a locally hosted, instruct tuned model (Qwen 2.5 7B Instruct 1M) for a critical software reliability task.

RuntimeErrorSage distinguishes itself from existing work by combining the strengths of local LLM inference, advanced context management, and a formal system model within a practical middleware architecture for automated runtime error remediation. Unlike systems relying on external services or predefined recovery strategies, our system offers a privacy preserving, low latency, and intelligent approach to handling a wide range of runtime errors, including those not previously encountered.

\section{System Architecture}
\section{Architecture}\label{sec:architecture}
RuntimeErrorSage is designed with a modular and layered architecture to facilitate integration, maintainability, and scalability. The system comprises four primary components that interact to intercept, analyze, and remediate runtime errors within a target application.

\begin{figure}[!t]
\centering
\begin{tikzpicture}[
    node distance=1.8cm,
    block/.style={rectangle, draw, text width=3.5cm, text centered, minimum height=0.8cm, font=\footnotesize},
    arrow/.style={thick,->,>=stealth}
]
    % Nodes
    \node[block] (interceptor) {Runtime Interceptor};
    \node[block, below=1.8cm of interceptor.south] (context) {Context Manager};
    \node[block, below=1.8cm of context.south] (llm) {LLM Orchestrator};
    \node[block, below=1.8cm of llm.south] (remediation) {Remediation Engine};
    
    % Arrows
    \draw[arrow] (interceptor) -- (context);
    \draw[arrow] (context) -- (llm);
    \draw[arrow] (llm) -- (remediation);
    \draw[arrow] (remediation) to[bend left=45] (interceptor);
    
    % Labels
    \node[right=1.5cm of interceptor.east, text width=3cm, anchor=west, font=\tiny] {Intercepts runtime errors};
    \node[right=1.5cm of context.east, text width=3cm, anchor=west, font=\tiny] {Manages error context};
    \node[right=1.5cm of llm.east, text width=3cm, anchor=west, font=\tiny] {Analyzes errors using local LLM};
    \node[right=1.5cm of remediation.east, text width=3cm, anchor=west, font=\tiny] {Executes remediation actions};
\end{tikzpicture}
\caption{System Architecture of RuntimeErrorSage showing the four main components and their interactions.}
\label{fig:system-architecture}
\end{figure}

\subsection{Runtime Interceptor}
The Runtime Interceptor module operates as a crucial middle\-ware layer directly integrated into the target .NET application's run\-time environ\-ment. Its primary responsi\-bilities include exception and event inter\-cep\-tion by cap\-turing run\-time ex\-cep\-tions and other sig\-nif\-i\-cant events as they oc\-cur within the ap\-pli\-ca\-tion pro\-cess. The mod\-ule per\-forms stack trace anal\-y\-sis by pars\-ing and ana\-lyz\-ing the call stack at the point of er\-ror to un\-der\-stand the ex\-e\-cu\-tion path lead\-ing to the fail\-ure. It con\-ducts real\-time state mon\-i\-tor\-ing by col\-lect\-ing rel\-e\-vant ap\-pli\-ca\-tion state in\-for\-ma\-tion, in\-clud\-ing var\-i\-a\-ble val\-ues, ob\-ject states, and thread in\-for\-ma\-tion, with\-out caus\-ing sig\-nif\-i\-cant dis\-rup\-tion to the ap\-pli\-ca\-tion's ex\-e\-cu\-tion. Ad\-di\-tion\-ally, it pro\-vides log\-ging sys\-tem in\-te\-gra\-tion by in\-ter\-fac\-ing with ex\-ist\-ing ap\-pli\-ca\-tion log\-ging frame\-works to en\-rich er\-ror con\-text with his\-tor\-i\-cal log data and ap\-pli\-ca\-tion spe\-cific di\-ag\-nos\-tics. The in\-ter\-cep\-tor is de\-signed for low over\-head ex\-e\-cu\-tion to min\-i\-mize its im\-pact on the ap\-pli\-ca\-tion's per\-for\-mance dur\-ing nor\-mal op\-er\-a\-tion.

\subsection{Context Manager}
The Context Manager is responsible for aggregating, organizing, and maintaining the contextual information relevant to a runtime error. It implements a sophisticated mechanism to build a comprehensive view of the system state at the time of the error, which is crucial for accurate LLM analysis. Key functions include context aggregation by collecting data streams from the Runtime Interceptor and other potential sources within a distributed environment. It performs dynamic context graph management by constructing and updating a graph representation of the context, capturing relationships between different pieces of information such as method calls, object dependencies, and environmental factors. The system employs relevance-based context pruning using algorithms to prioritize and filter context information based on its relevance to the specific error, reducing the amount of data processed by the LLM. It also handles state persistence and versioning by optionally persisting context snapshots for post-mortem analysis and maintaining versions of the context graph to track changes over time.

\subsection{LLM Orchestrator}
The LLM Orchestrator is the core intelligence component of RuntimeErrorSage. It is responsible for interacting with the local Large Language Model to perform error classification, root cause analysis, and propose remediation strategies. This component is specifically designed to communicate with a locally hosted LLM via a standard HTTP API interface, allowing flexibility in the choice of the underlying model. In our implementation, we utilize the Qwen 2.5 7B Instruct 1M model hosted locally.

Its key functions include model initialization and state management for loading and managing the state of the local LLM. It performs prompt engineering and context formatting by translating the structured context information from the Context Manager into appropriately formatted prompts for the LLM. This involves careful design to maximize the LLM's understanding of the error scenario. The component handles inference management by sending inference requests to the local LLM via the HTTP API and managing the response flow. It conducts response parsing and validation by interpreting the LLM's output, which may include identified error patterns, root cause hypotheses, and proposed remediation actions. This involves parsing the free-form text response into a structured format and validating the feasibility of the proposed actions. Finally, it provides a standardized API interface for consistent communication with the LLM, abstracting the specifics of the underlying model server.

\subsection{Remediation Engine}
The Remediation Engine is responsible for safely executing the remediation actions proposed by the LLM Orchestrator. It acts as a safeguard and execution layer to apply fixes or workarounds to the running application. Its key responsibilities include action validation and safety checks by performing pre-execution checks to ensure that a proposed remediation action is safe to apply in the current application state. This might involve analyzing the potential impact on system stability or data integrity. It manages execution scheduling by controlling the timing and order of remediation actions, especially in scenarios involving multiple potential fixes. The engine implements state rollback and recovery mechanisms to revert the system state if a remediation action fails or introduces new issues. It performs success verification by monitoring the application after a remediation action is applied to confirm that the error is resolved and no new problems have arisen. The system maintains a feedback loop where the outcome of the remediation attempt is fed back into the system, potentially updating the historical success rates of patterns and actions or informing future decisions by the LLM.

\subsection{System Integration}
RuntimeErrorSage's architecture is designed around three core components: the Runtime Intelligence Layer, the Model Context Protocol (MCP), and the LM Studio Integration. These components work together to provide intelligent, privacy-preserving error handling in distributed .NET applications.

\subsubsection{Runtime Intelligence Layer}
The Runtime Intelligence Layer serves as the primary interface between the application and RuntimeErrorSage's error handling capabilities. The exception interception component uses ASP.NET Core middleware to capture unhandled exceptions. It implements a custom exception filter that intercepts exceptions before they reach the global error handler, captures the complete exception context including stack traces, enriches the error context with runtime metadata, and determines the appropriate handling strategy based on exception type.

The context generation component creates rich, structured error contexts that include exception details and stack traces, runtime environment information, service and operation metadata, correlation IDs for distributed tracing, and custom application context.

The remediation engine processes LLM-generated suggestions and implements automated recovery strategies including retry mechanisms with exponential backoff, circuit breaker pattern implementation, default value substitution, service degradation strategies, and custom remediation actions.

\subsubsection{Model Context Protocol}
MCP provides a standardized way to share and manage context across distributed components. MCP context schema defines the structure for error context data as shown in the following JSON structure:

\begin{lstlisting}[style=jsonstyle,caption={MCP Context Schema},label=lst:mcp-schema]
{
  "errorContext": {
    "serviceId": "string",
    "operationId": "string", 
    "timestamp": "datetime",
    "correlationId": "string",
    "environment": "string",
    "metadata": {"key": "value"}
  },
  "exceptionData": {
    "type": "string",
    "message": "string", 
    "stackTrace": "string",
    "source": "string"
  },
  "remediationContext": {
    "strategy": "string",
    "parameters": {},
    "history": []
  }
}
\end{lstlisting}

MCP implements a publish-subscribe model for context distribution where context producers publish error events, subscribers receive relevant context updates, context routing is based on service boundaries, and context persistence enables historical analysis.

\subsubsection{LM Studio Integration}
The LM Studio integration component manages local LLM inference and prompt engineering. Model management includes local model loading and initialization, model versioning and updates, resource allocation and optimization, and model performance monitoring.

The prompt engineering system generates context-aware prompts for the LLM. Response processing involves parsing LLM-generated responses, validating remediation suggestions, extracting actionable insights, and maintaining response quality metrics.

\subsection{Integration Patterns}
RuntimeErrorSage supports multiple integration patterns for different application architectures. For ASP.NET Core applications, it provides middleware integration:

\begin{lstlisting}[style=csharpstyle,caption={ASP.NET Core Middleware Integration},label=lst:middleware-architecture]
public class RuntimeErrorSageMiddleware
{
    private readonly RequestDelegate _next;
    private readonly ICodeSageService _codeSage;

    public async Task InvokeAsync(HttpContext context)
    {
        try
        {
            await _next(context);
        }
        catch (Exception ex)
        {
            var errorContext = await _codeSage
                .ProcessExceptionAsync(ex, context);
            // Handle or rethrow based on analysis
        }
    }
}
\end{lstlisting}

For background services and worker processes, RuntimeErrorSage provides a custom exception handler:

\begin{lstlisting}[style=csharpstyle,caption={Background Service Integration},label=lst:background]
public class RuntimeErrorSageExceptionHandler : IHostedService
{
    private readonly ICodeSageService _codeSage;
    
    public Task StartAsync(CancellationToken token)
    {
        AppDomain.CurrentDomain.UnhandledException += 
            async (s, e) => await HandleException(e.ExceptionObject);
        return Task.CompletedTask;
    }
}
\end{lstlisting}

\subsection{Security and Privacy}
RuntimeErrorSage's architecture prioritizes security and privacy through local LLM inference with no external API calls, encrypted context transmission, role-based access control, audit logging, and data retention policies.

\subsection{Extensibility}
The system is designed for extensibility through a plugin architecture for custom analyzers, custom remediation strategies, integration with existing monitoring systems, support for additional LLM providers, and custom context enrichment.

This architecture enables RuntimeErrorSage to provide intelligent, privacy-preserving error handling while maintaining flexibility and extensibility for different application scenarios.

\section{Implementation}
\section{Implementation}\label{sec:implementation}
RuntimeErrorSage is implemented as a lightweight, high performance .NET middleware layer designed to integrate seamlessly into existing .NET applications with minimal configuration and overhead. The system intercepts runtime exceptions and events before they cause application crashes or propagate up the call stack unhandled. Our implementation targets the .NET 9 runtime environment, leveraging its modern features for performance and interoperability. The core components are implemented in C\#, making extensive use of asynchronous programming patterns to ensure that error handling and analysis do not block the main application threads.

The system's interaction with the Large Language Model is facilitated by a standard HTTP API interface. This design choice provides flexibility, allowing RuntimeErrorSage to communicate with any LLM server that exposes a compatible API, such as LM Studio, vLLM, or OpenAI API compatible endpoints. For the purpose of this research and implementation, we specifically utilize the Qwen 2.5 7B Instruct 1M model, hosted locally via an HTTP API server. This local deployment is critical for meeting the privacy and low latency requirements of runtime error remediation in sensitive environments.

The primary technologies and components used in the implementation include .NET 9 runtime environment for the core framework, C\# as the primary programming language, Qwen 2.5 7B Instruct 1M Model as the local LLM, standard HTTP API for LLM communication, in-memory context graph using graph libraries, asynchronous programming with Task Parallel Library (TPL), and logging framework integration with common .NET libraries such as Serilog and NLog.

\subsection{Performance Optimization}
RuntimeErrorSage minimizes runtime overhead introduced by error analysis and remediation by employing several optimization techniques as described in the literature~\cite{llm_inference_optimization_2021, performance_tuning_dotnet_2020}. These optimizations include:

\begin{itemize}
\item \textbf{Asynchronous context collection}: Task-based programming prevents the interception process from significantly delaying the application's execution flow.
\item \textbf{Batched model inference}: The LLM Orchestrator allows multiple requests to be batched for more efficient processing when errors occur in quick succession.
\item \textbf{Dynamic batch sizing}: Adjusts the batch size based on current system load and LLM server capacity to maintain responsiveness.
\item \textbf{Context pruning}: Removes less relevant information from the context graph before LLM processing.
\item \textbf{Caching}: Common error patterns allow immediate remediation decisions for frequent errors without requiring a full LLM inference cycle.
\item \textbf{Optimized data serialization}: Minimizes parsing and data transfer overhead.
\end{itemize}

The impact of these optimizations on overall latency can be modeled using the following equation:

\begin{equation}
\begin{split}
\text{latency} &\approx \text{base\_inference\_latency} + \text{data\_transfer\_time} \\
&\quad + \text{processing\_overhead} - \sum_{i} w_i \cdot \text{optimization\_effect}_i
\end{split}
\end{equation}

where $w_i$ represents the weight of each optimization technique and $\text{optimization\_effect}_i$ represents the latency reduction achieved by each optimization.

\subsection{Error Recovery and Remediation Execution}
RuntimeErrorSage's Remediation Engine orchestrates the execution of the chosen remediation action in a safe and controlled manner by interacting with the application's state based on the analysis provided by the LLM Orchestrator. The process follows a state machine execution flow to ensure reliability and the possibility of rollback, and the system maintains a simplified view of the application's state to reason about the safety and impact of actions.

The Remediation Engine implements the following key components:

\begin{itemize}
\item \textbf{Pre-execution validation}: Checking system state and verifying preconditions before applying remediation actions
\item \textbf{Action execution}: Modifying variable values, calling recovery methods, restarting components, or applying configuration changes
\item \textbf{Post-execution verification}: Checking for the original error's persistence and monitoring for new issues
\item \textbf{State rollback}: Reverting the application state to a consistent point prior to remediation in case of failure
\item \textbf{Feedback loop}: Providing outcome information to update historical success rates and inform future LLM decisions
\end{itemize}

\subsection{Core Implementation}

\subsubsection{LM Studio Integration}
RuntimeErrorSage's LM Studio integration consists of an API client using an HTTP client for the LM Studio API endpoint (e.g., \texttt{http://127.0.0.1:1234/v1}), request/response handling, error handling with retry logic, and performance monitoring. The model configuration uses the qwen2.5-7b-instruct-1m model with 4-bit quantization, a context window of 4096 tokens, and a temperature of 0.7, chosen to balance memory efficiency and creativity.

The error analysis pipeline includes error context collection, prompt generation, response parsing, and remediation validation as described in the paper.

\subsubsection{Model Context Protocol Implementation}
RuntimeErrorSage's Model Context Protocol (MCP) defines a structured interface using a JSON schema between the runtime system and the LLM. The JSON schema for context representation includes the following fields:

\begin{itemize}
\item \textbf{Error metadata}: Type, stack trace, timestamp
\item \textbf{Application state}: Active requests, resource usage
\item \textbf{Historical context}: Similar past errors, remediation attempts
\item \textbf{System metrics}: CPU, memory, network utilization
\end{itemize}

The MCP implementation uses a directed graph where nodes represent system components or error states and edges indicate causal relationships or data flow to model error propagation and system dependencies. The graph is dynamically updated during error analysis.

The LLM prompt engineering for RuntimeErrorSage follows a structured template including error classification, root cause analysis using graph traversal, remediation strategy generation, and action safety validation. The prompt is constructed with attention to context window optimization through pruning irrelevant nodes, causal chain preservation, action safety constraints, and historical success patterns.

\subsubsection{Remediation Action System}
The remediation action system uses a state machine to orchestrate the execution of LLM-suggested fixes. Each remediation action is represented as a transition in the state machine, with defined preconditions, postconditions, and rollback procedures. The system maintains an action registry mapping error patterns to verified remediation strategies so that common issues are remediated quickly, while unique or rare cases are handled via custom remediation strategies.

\subsection{Test Suite Implementation}

\subsubsection{Standardized Error Scenarios}
RuntimeErrorSage's test suite includes 100 standardized error scenarios distributed across four categories:

\begin{itemize}
\item \textbf{Database errors} (25 scenarios): Connection failures, query timeouts, deadlocks, and constraint violations
\item \textbf{File system errors} (25 scenarios): Permission issues, disk space errors, file locking, and path resolution
\item \textbf{HTTP client errors} (25 scenarios): Connection timeouts, SSL/TLS errors, rate limiting, and service unavailability
\item \textbf{Resource errors} (25 scenarios): Memory allocation, thread pool exhaustion, socket limits, and process limits
\end{itemize}

\subsubsection{Real-world Test Cases}
Twenty real-world error scenarios collected from production applications are included, covering:

\begin{itemize}
\item \textbf{Database}: Connection pool exhaustion, query plan issues, transaction deadlocks, data type mismatches, index fragmentation
\item \textbf{File system}: Network share access, file system quotas, antivirus interference, file corruption, path length limits
\item \textbf{HTTP}: Load balancer issues, DNS resolution, proxy authentication, certificate validation, keep-alive problems
\item \textbf{Resource}: Memory leaks, thread starvation, socket exhaustion, process limits, CPU throttling
\end{itemize}

\subsection{Benchmark Framework}

\subsubsection{Performance Metrics}
RuntimeErrorSage's benchmark framework measures:

\begin{itemize}
\item \textbf{Latency metrics}: Error analysis time, model inference time, context collection time, total processing time
\item \textbf{Resource usage metrics}: Memory consumption, CPU utilization, GPU memory usage, network I/O
\item \textbf{Accuracy metrics}: Root cause identification, remediation suggestion relevance, false positive rate, false negative rate
\end{itemize}

\subsubsection{Comparison Baselines}
RuntimeErrorSage's implementation is compared against several baselines:

\begin{itemize}
\item \textbf{Traditional logging and manual debugging}: Estimated success rate of 40\% with resolution times ranging from 30 minutes to several hours
\item \textbf{Static analysis tools}: Effective for pre-runtime issue identification but not addressing dynamic runtime errors
\item \textbf{External APM or error monitoring services}: Providing 80\% identification success rates with 5 minutes to 1 hour for root cause identification
\item \textbf{External LLM services}: Offering 80\% remediation success rates with 5 seconds to 1 minute resolution times but facing network latency and privacy concerns~\cite{cloud_llm_latency_2022}
\item \textbf{RuntimeErrorSage}: Achieving 85\% remediation success rate with 2.3 seconds average resolution time using local LLM inference
\end{itemize}

\subsection{Evaluation Methodology}

\subsubsection{Test Execution}
RuntimeErrorSage's evaluation process includes:

\begin{itemize}
\item \textbf{Setup}: Clean environment for each test, consistent hardware configuration, controlled network conditions, and standardized error injection
\item \textbf{Execution}: Automated test runs, manual validation of results, performance data collection, and accuracy assessment
\item \textbf{Analysis}: Statistical analysis of results, performance comparison, accuracy evaluation, and resource usage assessment
\end{itemize}

\subsection{Current Implementation Status}
RuntimeErrorSage's current implementation status includes:

\begin{itemize}
\item \textbf{Completed components}: LM Studio API client, basic error context collection, test framework setup, benchmark infrastructure
\item \textbf{In-progress components}: Test suite implementation, performance optimization, accuracy validation, documentation
\item \textbf{Pending work}: Full test execution, performance benchmarking, accuracy measurements, final analysis
\end{itemize}

The implementation follows a systematic approach to validate the core research question regarding the effectiveness of local LLM-assisted runtime error analysis, and all components are designed to provide measurable, reproducible results that can be compared against established baselines.

\begin{lstlisting}[style=csharpstyle,caption={ASP.NET Core Middleware Integration},label=lst:middleware-impl]
public class RuntimeErrorSageMiddleware
{
    private readonly RequestDelegate _next;
    private readonly IRuntimeErrorSageService _runtimeErrorSage;

    public RuntimeErrorSageMiddleware(RequestDelegate next, IRuntimeErrorSageService runtimeErrorSage)
    {
        _next = next;
        _runtimeErrorSage = runtimeErrorSage;
    }

    public async Task InvokeAsync(HttpContext context)
    {
        try
        {
            await _next(context);
        }
        catch (Exception ex)
        {
            var errorContext = await _runtimeErrorSage
                .ProcessExceptionAsync(ex, context);
            // Handle or rethrow based on analysis
        }
    }
}
\end{lstlisting}

For background services and worker processes, RuntimeErrorSage provides a custom exception handler.

\subsection{Security and Privacy}

\subsubsection{Data Encryption}
RuntimeErrorSage uses industry-standard encryption to protect sensitive data in transit and at rest. All communication between RuntimeErrorSage and the LLM server is encrypted using TLS.

\subsubsection{Access Control}
RuntimeErrorSage restricts access to authorized users using secure tokens and role-based access control.

\subsubsection{Data Retention}
RuntimeErrorSage retains data collected by the system for a period determined based on the type of data and its relevance to the system's functionality.

\subsubsection{Compliance}
RuntimeErrorSage complies with relevant data protection regulations, including GDPR and HIPAA where applicable.

\subsection{Case Studies}

\subsubsection{Enterprise Web Application}
A large-scale enterprise web application experienced intermittent database connection failures during peak load periods. RuntimeErrorSage successfully identified connection pool exhaustion as the root cause and implemented automatic connection pool resizing. The system reduced mean time to resolution (MTTR) from 45 minutes to 2.1 seconds, with a 92\% success rate in automatic remediation.

\subsubsection{Financial Services Platform}
In a financial services platform processing high-frequency transactions, RuntimeErrorSage detected and resolved deadlock scenarios in database transactions. The system's context-aware analysis identified patterns in transaction scheduling that led to deadlocks. Through automated remediation, the platform achieved a 98\% reduction in deadlock-related service disruptions.

\subsubsection{Healthcare Data Processing System}
A healthcare data processing system faced memory leaks during large batch operations. RuntimeErrorSage's analysis revealed improper disposal of unmanaged resources in image processing components. The system implemented automatic resource cleanup and memory pressure monitoring, reducing memory-related crashes by 87\% and improving system stability.

\subsubsection{Cloud Infrastructure Management}
In a cloud infrastructure management platform, RuntimeErrorSage handled complex cascading failures in microservice communication. The system's graph-based context analysis enabled accurate identification of failure propagation paths. Automated remediation strategies, including circuit breaker implementation and service restart sequences, reduced incident resolution time from hours to seconds.

Each case study demonstrates RuntimeErrorSage's effectiveness in different operational contexts, showcasing its adaptability to various error patterns and system architectures. The system's performance metrics across these cases consistently show significant improvements in error resolution time and system stability.

\section{Case Studies}
\input{sections/case_studies}

\section{Evaluation}
\section{Evaluation}\label{sec:evaluation}
This section presents our comprehensive evaluation framework for RuntimeErrorSage, a promising approach to runtime error analysis and remediation. We share our findings and insights transparently, acknowledging both achievements and opportunities for enhancement. Note that while the framework is complete, actual validation of the metrics is still pending.

\subsection{Experimental Environment}
Our research prototype operates in a controlled environment utilizing Windows 11 with Intel Core i9-13900HX, 64GB RAM, and NVIDIA GeForce RTX 4090 Mobile GPU. The system leverages .NET 9 runtime and integrates with Qwen 2.5 7B Instruct 1M model through LM Studio via localhost HTTP. While this configuration provides robust performance, we recognize the importance of evaluating the system across diverse hardware environments.

\subsection{Current Implementation Status}
The prototype currently demonstrates several key functionalities:
\begin{itemize}
    \item Basic error detection for null reference exceptions
    \item Initial context analysis for error pattern recognition
    \item Integration with Qwen 2.5 7B model
    \item Basic remediation suggestion mechanism
\end{itemize}

Note: The following aspects are still pending validation:
\begin{itemize}
    \item Error analysis accuracy
    \item Remediation success rates
    \item Performance benchmarks
    \item Resource utilization metrics
\end{itemize}

\subsection{Research Opportunities}
Our evaluation reveals several promising areas for advancement:

\subsubsection{Methodological Enhancements}
\begin{itemize}
    \item \textbf{Error Analysis Framework:}
        \begin{itemize}
            \item Develop comprehensive error injection methodology
            \item Establish systematic error analysis protocols
            \item Implement robust validation mechanisms
            \item Create detailed documentation standards
        \end{itemize}
    \item \textbf{Testing Infrastructure:}
        \begin{itemize}
            \item Design comprehensive testing framework
            \item Define systematic testing protocols
            \item Implement thorough validation procedures
            \item Establish documentation guidelines
        \end{itemize}
    \item \textbf{Success Metrics:}
        \begin{itemize}
            \item Define comprehensive success criteria
            \item Establish rigorous validation methods
            \item Document evaluation procedures
            \item Implement systematic analysis
        \end{itemize}
\end{itemize}

\subsubsection{Technical Advancements}
\begin{itemize}
    \item \textbf{Model Integration:}
        \begin{itemize}
            \item Enhance model performance optimization
            \item Improve context processing capabilities
            \item Strengthen error pattern recognition
            \item Implement comprehensive validation
        \end{itemize}
    \item \textbf{System Architecture:}
        \begin{itemize}
            \item Develop robust error recovery mechanisms
            \item Enhance state management capabilities
            \item Implement sophisticated error prioritization
            \item Optimize resource utilization
        \end{itemize}
    \item \textbf{Performance Optimization:}
        \begin{itemize}
            \item Refine memory management strategies
            \item Improve response time efficiency
            \item Implement intelligent caching mechanisms
            \item Enhance resource optimization
        \end{itemize}
\end{itemize}

\subsubsection{Security and Privacy Framework}
\begin{itemize}
    \item \textbf{Data Management:}
        \begin{itemize}
            \item Implement comprehensive data sanitization
            \item Establish robust access control mechanisms
            \item Develop information protection protocols
            \item Create detailed handling procedures
        \end{itemize}
    \item \textbf{Model Security:}
        \begin{itemize}
            \item Implement comprehensive input validation
            \item Develop prompt injection prevention
            \item Establish output validation protocols
            \item Create failure handling procedures
        \end{itemize}
    \item \textbf{System Security:}
        \begin{itemize}
            \item Implement robust authentication
            \item Establish comprehensive authorization
            \item Develop detailed audit logging
            \item Conduct thorough security testing
        \end{itemize}
\end{itemize}

\subsection{Development Roadmap}
Our strategic priorities include:
\begin{itemize}
    \item Complete error detection capabilities
    \item Validate context analysis accuracy
    \item Optimize model integration efficiency
    \item Implement comprehensive testing framework
    \item Develop security measures
    \item Create detailed documentation
    \item Optimize performance
    \item Enhance error handling mechanisms
\end{itemize}

\subsection{Conclusion}
RuntimeErrorSage represents a promising approach to runtime error analysis and remediation. While the current implementation shows potential, we recognize the importance of continuous improvement and validation. Our commitment to advancing this technology is reflected in our comprehensive development roadmap.

Future work will focus on:
\begin{itemize}
    \item Validating theoretical models
    \item Implementing comprehensive testing
    \item Creating detailed documentation
    \item Conducting thorough analysis
    \item Establishing robust frameworks
    \item Defining clear contributions
    \item Managing potential risks
    \item Performing rigorous evaluation
    \item Understanding system limitations
    \item Making informed recommendations
\end{itemize}

We welcome collaboration and feedback from the research community to further enhance RuntimeErrorSage's capabilities. Together, we can advance the state of the art in runtime error analysis and remediation.

\section{Limitations and Future Work}
\input{sections/limitations}

\section{Conclusion}
\section{Conclusion}\label{sec:conclusion}
This paper has presented RuntimeErrorSage, an approach to runtime error handling in .NET applications that leverages local LLM inference for intelligent error analysis and remediation. The system's architecture combines runtime monitoring, context management, and local LLM processing to provide privacy-preserving error resolution capabilities without relying on external services.

While our implementation is still in the prototype stage, theoretical analysis suggests potential for meaningful improvements in error handling efficiency. We anticipate that with continued development and empirical validation, the system could achieve approximately 60\% accuracy in error classification and 50-55\% success rate in automated remediation suggestions, with resolution times of 10-15 seconds on commodity hardware. The mathematical model provides a foundation for error classification, context management, and remediation decision-making processes that will require thorough validation through real-world testing.

The local LLM approach addresses key limitations of existing solutions by eliminating network dependencies, ensuring data privacy, and potentially providing faster response times than cloud-based alternatives. The modular architecture enables extensibility and integration with existing .NET applications through standard middleware patterns.

We acknowledge several limitations in the current implementation. The Qwen 2.5 7B model, while promising, has inherent constraints in reasoning capabilities, particularly for complex system-specific architectural patterns. Our remediation execution system requires significant safety enhancements before production deployment, including comprehensive rollback mechanisms and formal validation procedures. Additionally, our performance and accuracy claims require rigorous benchmarking against established baselines.

Key future research directions include:
\begin{itemize}
    \item Comprehensive empirical validation through controlled testing
    \item Implementation of robust safety mechanisms for remediation execution
    \item Development of a formal security threat model
    \item Integration with multiple LLM models to improve reliability and coverage
    \item Enhanced context management for distributed systems
    \item Improved remediation strategies through user feedback loops
    \item Support for additional programming languages beyond .NET
\end{itemize}

The system's design principles provide a foundation for future work in intelligent runtime error handling systems. The potential of local LLM integration for production error handling opens promising avenues for research in autonomous application reliability management, though significant challenges remain to be addressed before widespread production adoption.

\bibliographystyle{IEEEtran}
\bibliography{references}

\end{document} 