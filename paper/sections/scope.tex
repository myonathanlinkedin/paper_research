\section{Scope of Research}\label{sec:scope}

This research focuses on a single, well-defined contribution: evaluating the feasibility and effectiveness of local LLM-assisted runtime error analysis in .NET applications. The scope is deliberately limited to ensure proper validation and meaningful results.

\subsection{Core Research Question}
Can local LLM inference (via LM Studio) provide effective runtime error analysis and remediation suggestions in .NET applications, while maintaining privacy and performance requirements?

\subsection{Success Criteria}
The research will be considered successful if it can demonstrate:

\begin{itemize}
    \item \textbf{Error Analysis Accuracy:}
    \begin{itemize}
        \item At least 80\% accuracy in error root cause identification
        \item At least 70\% accuracy in remediation suggestion relevance
        \item Measured against a standardized test suite of common .NET errors
    \end{itemize}
    \item \textbf{Performance Requirements:}
    \begin{itemize}
        \item Error analysis latency under 500ms for 95\% of requests
        \item Memory overhead under 100MB for the LLM component
        \item CPU impact under 10\% during error analysis
    \end{itemize}
    \item \textbf{Implementation Completeness:}
    \begin{itemize}
        \item Fully functional LM Studio integration
        \item Complete test coverage of core components
        \item Documented API and integration patterns
    \end{itemize}
\end{itemize}

\subsection{Implementation Scope}
The implementation will be limited to:

\begin{itemize}
    \item \textbf{Core Components:}
    \begin{itemize}
        \item LM Studio integration with qwen2.5-7b-instruct-1m model
        \item Basic error context collection
        \item Standardized error response format
        \item Simple remediation execution
    \end{itemize}
    \item \textbf{Error Types:}
    \begin{itemize}
        \item Database connection errors
        \item File system errors
        \item HTTP client errors
        \item Resource allocation errors
    \end{itemize}
    \item \textbf{Application Types:}
    \begin{itemize}
        \item ASP.NET Core Web APIs
        \item Single-instance applications
        \item No distributed system requirements
    \end{itemize}
\end{itemize}

\subsection{Evaluation Methodology}
The research will be evaluated through:

\begin{itemize}
    \item \textbf{Test Suite:}
    \begin{itemize}
        \item 100 standardized error scenarios
        \item 20 real-world error cases
        \item Performance benchmark suite
        \item Memory usage analysis
    \end{itemize}
    \item \textbf{Comparison Baseline:}
    \begin{itemize}
        \item Traditional error handling (try-catch)
        \item Static analysis tools
        \item Manual debugging process
    \end{itemize}
    \item \textbf{Metrics:}
    \begin{itemize}
        \item Error resolution time
        \item Analysis accuracy
        \item System performance impact
        \item Memory usage
        \item CPU utilization
    \end{itemize}
\end{itemize}

\subsection{Out of Scope}
The following aspects are explicitly out of scope:

\begin{itemize}
    \item Distributed system error handling
    \item Advanced pattern recognition
    \item Custom LLM model training
    \item Complex remediation strategies
    \item Production deployment
    \item Security analysis
    \item Cross-platform support
\end{itemize}

\subsection{Implementation Status}
Current implementation status (as of [DATE]):

\begin{itemize}
    \item \textbf{Completed:}
    \begin{itemize}
        \item Basic error context collection
        \item LM Studio API integration
        \item Standardized error responses
        \item Test framework setup
    \end{itemize}
    \item \textbf{In Progress:}
    \begin{itemize}
        \item Error analysis accuracy validation
        \item Performance benchmarking
        \item Test suite implementation
        \item Documentation
    \end{itemize}
    \item \textbf{Pending:}
    \begin{itemize}
        \item Full test suite execution
        \item Performance optimization
        \item Final accuracy measurements
        \item Comparison with baselines
    \end{itemize}
\end{itemize}

\subsection{Timeline}
The research will be completed in the following phases:

\begin{itemize}
    \item \textbf{Phase 1 (Current):} Core Implementation
    \begin{itemize}
        \item Complete LM Studio integration
        \item Implement error context collection
        \item Develop test framework
        \item Create benchmark suite
    \end{itemize}
    \item \textbf{Phase 2:} Validation
    \begin{itemize}
        \item Execute test suite
        \item Measure accuracy
        \item Benchmark performance
        \item Compare with baselines
    \end{itemize}
    \item \textbf{Phase 3:} Documentation
    \begin{itemize}
        \item Document findings
        \item Analyze results
        \item Draw conclusions
        \item Identify limitations
    \end{itemize}
\end{itemize}

The research will be considered complete when all success criteria are met or when clear limitations are identified that prevent meeting the criteria. All results, including negative findings, will be documented and analyzed.