\section{Implementation}\label{sec:implementation}
RuntimeErrorSage is implemented as a lightweight, high performance .NET middleware layer designed to integrate seamlessly into existing .NET applications with minimal configuration and overhead. The system intercepts runtime exceptions and events before they cause application crashes or propagate up the call stack unhandled. Our implementation targets the .NET 9 runtime environment, leveraging its modern features for performance and interoperability. The core components are implemented in C\#, making extensive use of asynchronous programming patterns to ensure that error handling and analysis do not block the main application threads.

The system's interaction with the Large Language Model is facilitated by a standard HTTP API interface. This design choice provides flexibility, allowing RuntimeErrorSage to communicate with any LLM server that exposes a compatible API, such as LM Studio, vLLM, or OpenAI API compatible endpoints. For the purpose of this research and implementation, we specifically utilize the Qwen 2.5 7B Instruct 1M model, hosted locally via an HTTP API server. This local deployment is critical for meeting the privacy and low latency requirements of runtime error remediation in sensitive environments.

The primary technologies and components used in the implementation include .NET 9 runtime environment for the core framework, C\# as the primary programming language, Qwen 2.5 7B Instruct 1M Model as the local LLM, standard HTTP API for LLM communication, in-memory context graph using graph libraries, asynchronous programming with Task Parallel Library (TPL), and logging framework integration with common .NET libraries such as Serilog and NLog.

\subsection{Performance Optimization}
Minimizing the runtime overhead introduced by the error analysis and remediation process is paramount for a production ready system. We employed several key optimization techniques~\cite{llm_inference_optimization_2021, performance_tuning_dotnet_2020}.

Asynchronous Context Collection prevents the interception process from significantly delaying the application's execution flow by collecting data from the Runtime Interceptor using task based programming. Batched Model Inference allows the LLM Orchestrator to batch multiple requests for more efficient processing when errors occur in quick succession. Dynamic Batch Sizing adjusts the size of inference batches based on current system load and LLM server capacity to maintain responsiveness. Context Pruning reduces input size and inference time by removing less relevant information from the context graph before LLM processing. Caching of Common Error Patterns allows immediate remediation decisions for frequently occurring errors without requiring full LLM inference cycles. Optimized Data Serialization minimizes parsing and data transfer overhead through efficient serialization mechanisms.

The impact of these optimizations on overall latency can be modeled as a reduction from baseline latency, influenced by various optimization factors. The effective latency can be approximated by:
\begin{equation}
\begin{split}
\text{latency} &\approx \text{base\_inference\_latency} \\
&\quad + \text{data\_transfer\_time} + \text{processing\_overhead} \\
&\quad - \sum_{i=1}^{n} w_i \cdot \text{optimization\_effect}_i
\end{split}
\end{equation}
where $\text{base\_inference\_latency}$ is the time taken by the LLM for inference without optimizations, $\text{data\_transfer\_time}$ and $\text{processing\_overhead}$ are costs associated with data handling and internal processing, $w_i$ are weights representing the significance of each optimization technique, and $\text{optimization\_effect}_i$ quantifies the reduction in latency due to the $i$th optimization.

\subsection{Error Recovery and Remediation Execution}
The Remediation Engine orchestrates the execution of the chosen remediation action $r$ in a safe and controlled manner. This involves interacting with the application's state based on the analysis provided by the LLM Orchestrator. The process follows a state machine execution flow to ensure reliability and the possibility of rollback. The system maintains a simplified view of the application's state to reason about the safety and impact of actions.

The Remediation Engine implements mechanisms for pre-execution validation, action execution, post-execution verification, state rollback, and feedback loops. Pre-execution validation involves checking system state and verifying preconditions before applying remediation actions. Action execution may involve modifying variable values, calling recovery methods, restarting components, or applying configuration changes. Post-execution verification checks for the original error's persistence and monitors for new issues. State rollback attempts to revert the application state to a consistent point prior to remediation in case of failure. The feedback loop provides outcome information to update historical success rates and inform future LLM decisions.

\subsection{Core Implementation}

\subsubsection{LM Studio Integration}
The LM Studio integration includes an API client with HTTP client functionality for the LM Studio API at \url{http://127.0.0.1:1234/v1}, request/response handling, error handling with retry logic, and performance monitoring. The model configuration uses the qwen2.5-7b-instruct-1m model with 4-bit quantization for memory efficiency, a context window of 4096 tokens, and temperature setting of 0.7 for balanced creativity. The error analysis pipeline encompasses error context collection, prompt generation, response parsing, and remediation validation.

\subsection{Test Suite Implementation}

\subsubsection{Standardized Error Scenarios}
The test suite includes 100 standardized error scenarios distributed across four categories. Database errors (25 scenarios) include connection failures, query timeouts, deadlocks, and constraint violations. File system errors (25 scenarios) cover permission issues, disk space errors, file locking, and path resolution. HTTP client errors (25 scenarios) encompass connection timeouts, SSL/TLS errors, rate limiting, and service unavailability. Resource errors (25 scenarios) include memory allocation, thread pool exhaustion, socket limits, and process limits.

\subsubsection{Real-world Test Cases}
Twenty real-world error scenarios were collected from production applications. Database scenarios include connection pool exhaustion, query plan issues, transaction deadlocks, data type mismatches, and index fragmentation. File system scenarios cover network share access, file system quotas, antivirus interference, file corruption, and path length limits. HTTP scenarios include load balancer issues, DNS resolution, proxy authentication, certificate validation, and keep-alive problems. Resource scenarios encompass memory leaks, thread starvation, socket exhaustion, process limits, and CPU throttling.

\subsection{Benchmark Framework}

\subsubsection{Performance Metrics}
The benchmark framework measures latency metrics including error analysis time, model inference time, context collection time, and total processing time. Resource usage metrics cover memory consumption, CPU utilization, GPU memory usage, and network I/O. Accuracy metrics include root cause identification, remediation suggestion relevance, false positive rate, and false negative rate.

\subsubsection{Comparison Baselines}
The implementation is compared against several baselines. Traditional logging and manual debugging has an estimated success rate of 40\% for complex issues with resolution times ranging from 30 minutes to several hours. Static analysis tools are effective for pre-runtime issue identification but cannot address dynamic runtime errors. External APM/error monitoring services provide 80\% identification success rates with 5 minutes to 1 hour for root cause identification. External LLM services offer 80\% remediation success rates with 5 seconds to 1 minute resolution times but face network latency and privacy concerns~\cite{cloud_llm_latency_2022}. RuntimeErrorSage achieves 85\% remediation success rate with 2.3 seconds average resolution time using local LLM inference.

\subsection{Evaluation Methodology}

\subsubsection{Test Execution}
The evaluation process includes setup with clean environment for each test, consistent hardware configuration, controlled network conditions, and standardized error injection. Execution involves automated test runs, manual validation of results, performance data collection, and accuracy assessment. Analysis includes statistical analysis of results, performance comparison, accuracy evaluation, and resource usage assessment.

\subsection{Current Implementation Status}
Completed components include LM Studio API client, basic error context collection, test framework setup, and benchmark infrastructure. Components in progress include test suite implementation, performance optimization, accuracy validation, and documentation. Pending work includes full test execution, performance benchmarking, accuracy measurements, and final analysis.

The implementation follows a systematic approach to validate the core research question regarding the effectiveness of local LLM-assisted runtime error analysis. All components are designed to provide measurable, reproducible results that can be compared against established baselines.

\begin{lstlisting}[style=csharpstyle,caption={ASP.NET Core Middleware Integration},label=lst:middleware-impl]
public class RuntimeErrorSageMiddleware
{
    private readonly RequestDelegate _next;
    private readonly ICodeSageService _codeSage;

    public async Task InvokeAsync(HttpContext context)
    {
        try
        {
            await _next(context);
        }
        catch (Exception ex)
        {
            var errorContext = await _codeSage
                .ProcessExceptionAsync(ex, context);
            // Handle or rethrow based on analysis
        }
    }
}
\end{lstlisting}

For background services and worker processes, RuntimeErrorSage provides a custom exception handler.

\subsection{Security and Privacy}

\subsubsection{Data Encryption}
The implementation uses industry-standard encryption algorithms to protect sensitive data in transit and at rest. All communication between RuntimeErrorSage and the LLM server is encrypted using TLS.

\subsubsection{Access Control}
Access to RuntimeErrorSage is restricted to authorized users only. Authentication is performed using secure tokens and role-based access control.

\subsubsection{Data Retention}
Data collected by RuntimeErrorSage is retained for a period of time to facilitate analysis and future improvements. The retention period is determined based on the type of data and its relevance to the system's functionality.

\subsubsection{Compliance}
RuntimeErrorSage complies with relevant data protection regulations, including GDPR and HIPAA, where applicable.