\section{Evaluation}\label{sec:evaluation}
To rigorously assess the effectiveness and performance of RuntimeErrorSage, we conducted a comprehensive evaluation using a diverse test suite of runtime errors in .NET applications. The evaluation aims to quantify the system's ability to accurately classify errors, successfully remediate them, and operate with minimal overhead. Our evaluation framework builds upon established methodologies for evaluating error handling and self-healing systems \cite{software_dependability_evaluation_2014, self_healing_systems_metrics_2018}.

\subsection{Experimental Setup}
Our experimental setup was designed to simulate realistic application environments and error conditions. We utilized a dedicated Windows 11 development machine with Intel Core i9-13900HX processor, 64GB RAM, NVIDIA GeForce RTX 4090 Mobile GPU, SSD storage, .NET 9 runtime, and LM Studio hosting the Qwen 2.5 7B Instruct 1M model accessed via HTTP on localhost.

We constructed a comprehensive test suite comprising 100 common .NET runtime errors including frequently encountered exceptions such as NullReferenceException, IndexOutOfRangeException, DivideByZeroException, FileNotFoundException, NetworkException, and various argument exceptions. These errors were programmatically injected into a sample .NET application designed to mimic typical application structures. Additionally, we created 50 custom application errors to test the system's ability to handle application-specific logic errors and exceptions, including scenarios involving errors related to data validation, business rule violations, and interactions with mock external services. For both common and custom errors, we varied the depth of the call stack, the number and complexity of in-scope variables, and the amount of recent log data to evaluate the system's performance under different contextual loads.

The sample application used for injecting errors was a multi-threaded web application simulating typical data processing and user interaction patterns, allowing us to assess the system's behavior in a concurrent environment.

\subsection{Performance Metrics}
We measured the performance of RuntimeErrorSage using several key metrics widely accepted in the evaluation of reliability and self-healing systems. Error classification accuracy assesses the Error Classification Model's ability to correctly identify the pattern or root cause of a given error:

\begin{equation}
\label{eq:accuracy}
\text{Accuracy} = \frac{\text{Number of Correctly Classified Errors}}{\text{Total Number of Errors Tested}} \times 100\%
\end{equation}

Remediation success rate measures the Remediation Engine's ability to apply a chosen action that resolves the original error without introducing new issues:

\begin{equation}
\label{eq:success_rate}
\text{Success Rate} = \frac{\text{Successfully Remediated Errors}}{\text{Total Remediation Attempts}} \times 100\%
\end{equation}

Mean Time to Resolution (MTTR) measures the average time taken from error detection to successful resolution:

\begin{equation}
\label{eq:mttr}
\text{MTTR} = \frac{\sum_{e \in \mathcal{E}_{resolved}} (t_{resolved}(e) - t_{detected}(e))}{|\mathcal{E}_{resolved}|}
\end{equation}

where $\mathcal{E}_{resolved}$ is the set of successfully resolved errors, $t_{detected}(e)$ is the error detection time, and $t_{resolved}(e)$ is the resolution time.

Runtime overhead quantifies the performance impact of running RuntimeErrorSage as middleware:

\begin{equation}
\label{eq:overhead}
\text{Overhead} = \frac{T_{with} - T_{without}}{T_{without}} \times 100\%
\end{equation}

where $T_{with}$ and $T_{without}$ represent application execution times with and without RuntimeErrorSage respectively.

\subsection{Results}
Our experimental evaluation yielded promising results, demonstrating the effectiveness of RuntimeErrorSage in intelligently analyzing and remediating runtime errors locally. The system achieved 92\% average accuracy in classifying the tested runtime errors, indicating that the combination of context analysis and local LLM inference is highly effective in identifying the correct error pattern or root cause.

RuntimeErrorSage demonstrated an 85\% success rate in automatically remediating errors for which a remediation action was attempted. This highlights the LLM's ability to propose effective fixes and the Remediation Engine's capability to apply them safely. The Mean Time to Resolution (MTTR) was measured at an average of 2.3 seconds, representing a significant improvement over traditional debugging approaches. The measured runtime overhead introduced by RuntimeErrorSage during normal application execution was less than 5\%, demonstrating that the system's performance optimization techniques are effective in minimizing impact on the target application.

\subsection{Comparison with Existing Solutions}
We compared RuntimeErrorSage against representative existing approaches for error handling and analysis, drawing upon published benchmarks and characteristics of these systems \cite{appinsights_performance_2021, newrelic_overhead_2022, traditional_debugging_cost_2019}.

Traditional logging and manual debugging approaches are highly dependent on human expertise and log message quality, with estimated success rates of 40\% for complex issues and resolution times ranging from 30 minutes to several hours. Static analysis tools are effective at identifying potential pre-runtime issues but cannot address dynamic runtime errors or provide runtime remediation capability.

External APM and error monitoring services such as AppInsights and NewRelic provide detailed error reporting and some automated analysis but typically do not offer automated runtime remediation and involve data egress concerns. These services achieve approximately 80\% success rates for root cause identification with resolution times ranging from 5 minutes to 1 hour.

External LLM services using cloud-hosted models can offer intelligent analysis and potential remediation strategies but are subject to network latency, privacy concerns, and cost per inference \cite{cloud_llm_latency_2022}. These services achieve estimated success rates of 80\% for remediation with resolution times of 5 seconds to 1 minute depending on network conditions and model performance.

In contrast, RuntimeErrorSage achieved an 85\% remediation success rate with an average resolution time of 2.3 seconds while maintaining complete data privacy through local processing. This comparison highlights the advantage of RuntimeErrorSage's local LLM approach in achieving a balance of high remediation success rate and very low resolution time while addressing privacy concerns inherent in traditional cloud-based solutions.