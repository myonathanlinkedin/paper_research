\section{Related Work}\label{sec:related-work}
Recent advances in Large Language Models (LLMs) have revolutionized software development practices.

\subsection{Runtime Error Analysis}
Traditional approaches to runtime error analysis primarily rely on manual inspection of logs and debugging tools, static code analysis to identify potential issues before runtime, and post mortem analysis of crash dumps~\cite{debugging_techniques_survey_2017, static_analysis_overview_2015}. While effective for certain types of errors, these methods often struggle with dynamic runtime phenomena, complex interactions in distributed systems, and require significant human effort and expertise. 

Automated log analysis techniques~\cite{log_analysis_survey_2016} have been developed to process large volumes of log data, but they typically depend on predefined patterns and lack the ability to reason about novel error scenarios or system specific context without explicit programming.

\subsection{Self Healing Systems}
Research into self healing or autonomic computing systems has explored architectures and mechanisms for software systems to detect, diagnose, and recover from failures autonomously~\cite{autonomic_computing_overview_2004, self_healing_survey_2012}. These systems often employ feedback loops, such as the Monitor Analyze Plan Execute Knowledge (MAPE-K) loop~\cite{mapek2003}, to manage their own behavior and adapt to changing conditions or failures.

Remediation strategies in these systems can range from simple restarts and reconfigurations to more complex state rollbacks or dynamic code updates. However, many existing self healing solutions require significant a priori knowledge about potential failure modes and corresponding recovery actions, limiting their effectiveness against unforeseen errors. The integration of AI techniques, particularly machine learning, has been explored to enhance the diagnostic and planning capabilities of self healing systems, but leveraging the natural language understanding and reasoning abilities of LLMs for complex error scenarios represents a newer direction.

\subsection{Context Aware Computing and Debugging}
Context aware computing focuses on systems that can perceive their environment and adapt their behavior based on contextual information~\cite{context_aware_computing_survey_2009}. In the realm of software engineering and debugging, context awareness involves utilizing information about the system's state, execution environment, user interactions, and history to aid in understanding and resolving issues~\cite{context-aware-debugging-2023}.

Techniques include dynamic slicing, state tracing, and environmental monitoring to gather relevant context. While these techniques are powerful for providing visibility into the system, the challenge remains in effectively processing and reasoning about potentially vast and complex contextual data to pinpoint the root cause of an error and devise an appropriate solution. Our work utilizes context management techniques but enhances the analysis capabilities by feeding this context into a powerful LLM.

\subsection{Large Language Models in Software Engineering}
Large Language Models have rapidly emerged as powerful tools for a variety of software engineering tasks, including code completion~\cite{copilot2021}, code generation~\cite{llm_code_generation_2022}, code summarization, and vulnerability detection~\cite{llm_security_applications_2023}. Their ability to understand and generate human language and code has opened possibilities for more intelligent automated tools.

However, directly applying general purpose LLMs to real time, performance critical tasks like runtime error remediation requires careful consideration of latency, cost, and data privacy. The use of smaller, specialized, or locally hosted models is an active area of research to address these challenges~\cite{local_llm_deployment_2023, edge_llm_inference_2022}. Our approach specifically investigates the practical application of a locally hosted, instruct tuned model (Qwen 2.5 7B Instruct 1M) for a novel and critical software reliability task.

RuntimeErrorSage distinguishes itself from existing work by combining the strengths of local LLM inference, advanced context management, and a formal system model within a practical middleware architecture for automated runtime error remediation. Unlike systems relying on external services or predefined recovery strategies, our system offers a privacy preserving, low latency, and intelligent approach to handling a wide range of runtime errors, including those not previously encountered.