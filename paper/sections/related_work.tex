\subsection{AI-Assisted Programming and Debugging}
Recent advances in Large Language Models (LLMs) have revolutionized software development practices. Tools like GitHub Copilot \cite{copilot2021} and Amazon CodeWhisperer \cite{codewhisperer2022} have demonstrated the potential of AI-assisted programming, primarily focusing on code generation and completion. However, these tools operate primarily during development time, leaving a gap in runtime error handling and debugging assistance.

Runtime error analysis has traditionally relied on static analysis tools such as SonarQube \cite{sonarqube2023} and static analyzers like Roslyn \cite{roslyn2015}. While effective for detecting potential issues during development, these tools cannot address runtime-specific errors that emerge during execution. Dynamic analysis tools like Application Insights \cite{appinsights2023} and New Relic \cite{newrelic2023} provide runtime monitoring but lack intelligent error analysis and automated remediation capabilities.

\subsection{Runtime Error Handling and Self-Healing Systems}
Self-healing systems have been an active area of research, with approaches ranging from simple retry mechanisms to complex autonomic computing systems. The MAPE-K (Monitor-Analyze-Plan-Execute-Knowledge) loop \cite{mapek2003} has been widely adopted in autonomic computing, but its application in runtime error handling has been limited by the complexity of accurate error analysis and remediation planning.

Microservice architectures have introduced new challenges in error handling, with circuit breakers \cite{hystrix2012} and bulkheads \cite{bulkhead2014} becoming common patterns. However, these patterns often require manual configuration and lack intelligent adaptation to specific error contexts. Recent work in chaos engineering \cite{chaos2017} has focused on resilience testing but does not address real-time error analysis and remediation.

\subsection{Distributed Context Management}
The Model Context Protocol (MCP) builds upon existing context management approaches in distributed systems. OpenTelemetry \cite{opentelemetry2023} provides standardized instrumentation and observability, while distributed tracing systems like Jaeger \cite{jaeger2019} and Zipkin \cite{zipkin2012} focus on request tracing. However, these systems primarily serve monitoring and debugging purposes rather than enabling intelligent error handling.

Context-aware computing has been explored in various domains, from mobile computing \cite{contextaware2001} to IoT systems \cite{iotcontext2018}. The Context Toolkit \cite{contexttoolkit2001} introduced a framework for context management, but its application in runtime error handling has been limited. Recent work in context-aware middleware \cite{contextmiddleware2022} has shown promise in managing distributed context but lacks integration with AI-driven analysis.

\subsection{Local LLM Inference and Privacy}
The emergence of local LLM inference through tools like LM Studio \cite{lmstudio2023} and Ollama \cite{ollama2023} has enabled privacy-preserving AI applications. This approach addresses concerns about data privacy and connectivity requirements in production environments. However, the application of local LLMs in runtime error handling remains largely unexplored.

Federated learning \cite{federated2017} and edge computing \cite{edge2018} have demonstrated the benefits of local processing, but their integration with runtime error handling systems has been limited. The combination of local LLM inference with distributed context management, as proposed in CodeSage, represents a novel approach to privacy-preserving runtime intelligence.

\subsection{Research Gaps and Opportunities}
The review of existing work reveals several gaps that CodeSage aims to address:

\begin{itemize}
    \item The lack of intelligent, real-time error analysis in production environments
    \item Limited integration between context management and AI-driven error handling
    \item Absence of privacy-preserving approaches to runtime error analysis
    \item Need for standardized protocols for distributed error context sharing
    \item Limited automation in error remediation strategies
\end{itemize}

CodeSage's novel approach combines local LLM inference with standardized context management to provide intelligent, privacy-preserving runtime error handling. By leveraging the LM Studio API and MCP, it addresses these gaps while maintaining the benefits of existing approaches in distributed systems and error handling. 