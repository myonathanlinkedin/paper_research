\section{Introduction}
Modern software applications, especially complex and distributed systems, face significant challenges in effectively handling runtime errors~\cite{dist_systems_errors_2018, microservices_challenges_2020, runtime_error_analysis_2023, error_handling_survey_2024}. Traditional error management strategies, relying primarily on static analysis, detailed logging, and manual debugging, are often insufficient to address the dynamic and intricate nature of errors encountered in production environments~\cite{debugging_complex_systems_2021, runtime_debugging_challenges_2023, error_recovery_survey_2024}. These methods frequently lead to prolonged downtime, increased operational costs, and a suboptimal user experience due to delayed error identification and resolution.

Recent advancements in Large Language Models (LLMs) have demonstrated remarkable capabilities in understanding and generating code, opening new avenues for automated software engineering tasks, including code analysis and debugging support~\cite{llm_code_generation_2022, llm_debugging_2023, llm_software_engineering_2024, local_llm_applications_2024}. However, applying large, powerful LLMs directly to real time runtime error analysis in sensitive or resource constrained environments presents its own set of challenges. Privacy concerns associated with transmitting potentially sensitive runtime data to external services, the need for low latency responses for real time remediation, and the dependency on stable network connectivity limit the applicability of cloud hosted LLMs in many scenarios~\cite{privacy_llm_challenges_2023, edge_ai_challenges_2019, local_llm_security_2024, runtime_llm_optimization_2024}.

RuntimeErrorSage addresses these critical limitations by proposing and implementing a runtime middleware system that leverages a local Large Language Model for intelligent error analysis and automated remediation. Operating entirely offline, RuntimeErrorSage utilizes a standard HTTP API interface to interact with a locally hosted LLM, specifically the Qwen 2.5 7B Instruct 1M model. This approach ensures data privacy, minimizes latency, and provides a robust solution independent of external network dependencies, making it particularly suitable for enterprise applications, edge deployments, and environments with strict data governance policies.

Our work makes the following key contributions:
\begin{itemize}
\item We introduce RuntimeErrorSage, a system architecture for intelligent runtime error analysis and automated remediation utilizing a local LLM.
\item We present a formal mathematical framework encompassing models for runtime error classification, context management, and remediation decision making.
\item We detail the implementation of a .NET middleware layer for real time error interception and processing.
\item We provide a comprehensive evaluation demonstrating the system's effectiveness in terms of error classification accuracy, remediation success rate, and runtime overhead.
\item We show that leveraging a local, instruct tuned LLM (Qwen 2.5 7B Instruct 1M) via a standard API enables performant and privacy preserving runtime error handling.
\item The integration of AI techniques, particularly machine learning, has been explored to enhance the diagnostic and planning capabilities of self healing systems, but leveraging the natural language understanding and reasoning abilities of LLMs for complex error scenarios represents a newer direction.
\end{itemize}

RuntimeErrorSage distinguishes itself from existing work by combining the strengths of local LLM inference, advanced context management, and a formal system model within a practical middleware architecture for automated runtime error remediation. Unlike systems relying on external services or predefined recovery strategies, our system offers a privacy preserving, low latency, and intelligent approach to handling a wide range of runtime errors, including those not previously encountered.

The current implementation of RuntimeErrorSage includes a robust exception handling system with context-aware error tracking, ASP.NET Core middleware for exception interception, and standardized error response models. The system demonstrates practical capabilities in handling common runtime errors through example endpoints for database operations, file management, service integration, and resource allocation.

RuntimeErrorSage's architecture is designed to intercept unhandled exceptions during application execution, generate rich contextual information, and leverage local LLM inference to provide natural language explanations and remediation suggestions. The system operates fully offline, addressing critical privacy and connectivity constraints while maintaining interoperability through the MCP framework.

The source code for RuntimeErrorSage, including the implementation of the middleware layer, LM Studio integration, and Model Context Protocol, is available as open-source software at \url{https://github.com/myonathanlinkedin/paper_research}.

The remainder of this paper is organized as follows: Section~\ref{sec:related-work} reviews related work in AI-assisted programming, static analysis, and runtime error handling. Section~\ref{sec:scope} presents the scope of this research. Section~\ref{sec:implementation} describes the implementation details, including the middleware components and LLM integration. Section~\ref{sec:case-studies} presents case studies demonstrating RuntimeErrorSage's effectiveness. Section~\ref{sec:evaluation} evaluates the system's performance and accuracy. Section~\ref{sec:conclusion} discusses limitations and concludes the paper.